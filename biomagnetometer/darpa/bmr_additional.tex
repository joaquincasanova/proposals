
\begin{description}
\item [BAA number] HR001117S0025
\item [Lead Organization] Jenshan Lin, University of Florida
\item [Type of organization] Other Educational
\item [Proposer’s internal reference number]
\item [Other team members]:
  \begin{description}
  \item [Changzhi Li] Texas Tech University
  \item [YK Yoon] University of Florida 
  \item [Joaquin Casanova] University of Florida 
  \item [Proposal title] Biomimetic microfabricated magnetic gradiometer
  \item [Administrative PoC]:
    \begin{description}
    \item [Steven Blodgett]
    \item [1064 Center Dr, 216 LAR, Gainesville, FL 32611]
    \item [352-846-3948, blodgetts@ece.ufl.edu]
    \end{description}
  \item [Technical PoC]:
    \begin{description}
    \item [Casanova, Joaquin]
    \item [1064 Center Dr, 565 NEB, Gainesville, FL 32611]
    \item [352-246-9649, jcasa@ufl.edu]
    \end{description}
  \end{description}
\item [Total funds requested]
\item [Submitted] 06/01/2017
\item [Award instrument requested] Cost-Plus-Fixed Fee (CPFF), Cost-contract—no fee, cost sharing contract—no fee,or other type of procurement contract (specify) or Other Transaction
\item [Place(s) and period(s) of performance:] Univeristy of Florida, Texas Tech University, 10/01/2017-4/01/2021
\item [Total proposed cost] separated by basic award and option(s), if any, by calendar year
and by government fiscal year
\item [Defense Contract Management Agency (DCMA) administration office (if known)] Name, address, and telephone number
\item [Defense Contract Audit Agency (DCAA) audit office (if known)] Name, address, and telephone number
\item [Date proposal was prepared]
\item [DUNS]
\item [TIN]
\item [CAGE]
\item [Subcontractor Information]
\item [Proposal validity period] (120 days is recommended)
\item [Any Forward Pricing Rate Agreement] other such approved rate information, or such documentation that may assist in expediting negotiations (if available).
Attachment 1, the Cost Volume Proposer Checklist, must be included with the coversheet
of the Cost Proposal.
\end{description}
\newpage

\required{Detailed Cost Information (Prime and Subcontractors)}
The proposer’s (to include FFRDCs and Government Labs) cost volume shall provide cost and
pricing information, or other than cost or pricing information if the total price is under the
referenced threshold (See Note 1), in sufficient detail to substantiate the program price proposed
(e.g., realism and reasonableness). In doing so, the proposer shall provide, for both the prime
and each subcontractor, a “Summary Cost Breakdown” by phase and performer fiscal year,
and a “Detailed Cost Breakdown” by phase, technical task/sub-task, and month. The
breakdown/s shall include, at a minimum, the following major cost item along with associated
backup documentation:
Total program cost broken down by major cost items:

\section{Direct Labor}
A breakout clearly identifying the individual labor categories with associated labor hours and
direct labor rates, as well as a detailed Basis-of-Estimate (BOE) narrative description of the
methods used to estimate labor costs
\section{Indirect Costs}
Including Fringe Benefits, Overhead, General and Administrative Expense, Cost of Money,
Fee, etc. (must show base amount and rate)
\section{Travel}
Provide the purpose of the trip, number of trips, number of days per trip, departure and
arrival destinations, number of people, etc. See Section IV.B.13 for travel funding
restrictions
\section{Other Direct Costs}
Itemized with costs; back-up documentation is to be submitted to support proposed costs
\section{Material/Equipment}
(i) For IT and equipment purchases, include a letter stating why the proposer cannot provide
the requested resources from its own funding.
(ii) A priced Bill of Material (BOM) clearly identifying, for each item proposed, the
quantity, unit price, the source of the unit price (i.e., vendor quote, engineering estimate,
etc.), the type of property (i.e., material, equipment, special test equipment, information
technology, etc.), and a cross-reference to the Statement of Work (SOW) task/s that require
the item/s. At time of proposal submission, any item with a unit price that exceeds \$1,000
must be supported with basis-of-estimate (BOE) documentation such as a copy of catalog
price lists, vendor quotes or a detailed written engineering estimate (additional
documentation may be required during negotiations, if selected).
(iii) If seeking a procurement contract and items of Contractor Acquired Property are
proposed, exclusive of material, the proposer shall clearly demonstrate that the inclusion of
such items as Government Property is in keeping with the requirements of FAR Part 45.102.
In accordance with FAR 35.014, “Government property and title,” it is the Government’s
intent that title to all equipment purchased with funds available for research under any
resulting contract will vest in the acquiring nonprofit institution (e.g., Nonprofit Institutions
of Higher Education and Nonprofit Organizations whose primary purpose is the conduct of
scientific research) upon acquisition without further obligation to the Government. Any such
equipment shall be used for the conduct of basic and applied scientific research. The above
transfer of title to all equipment purchased with funds available for research under any
resulting contract is not allowable when the acquiring entity is a for-profit organization;
however, such organizations can, in accordance with FAR 52.245-1(j), be given priority to
acquire such property at its full acquisition cost.
\section{Consultants}
If consultants are to be used, proposer must provide a copy of the consultant’s proposed
SOW as well as a signed consultant agreement or other document which verifies the
proposed loaded daily / hourly rate and any other proposed consultant costs (e.g. travel);
\section{Subcontracts}
Itemization of all subcontracts. Additionally, the prime contractor is responsible for
compiling and providing, as part of its proposal submission to the Government, subcontractor
proposals prepared at the same level of detail as that required by the prime. Subcontractor
proposals include Interdivisional Work Transfer Agreements (ITWA) or similar
arrangements. If seeking a procurement contract, the prime contractor shall provide a cost
reasonableness analysis of all proposed subcontractor costs/prices. Such analysis shall
indicate the extent to which the prime contractor has negotiated subcontract costs/prices and
whether any such subcontracts are to be placed on a sole-source basis.
All proprietary subcontractor proposal documentation (fully disclosed subcontract proposal),
prepared at the same level of detail as that required of the prime, which cannot be uploaded
to the DARPA BAA website (https://baa.darpa.mil, BAAT) as part of the proposer’s
submission, shall be made immediately available to the Government, upon request, under
separate cover (i.e., mail, electronic/email, etc.), either by the proposer or by the
subcontractor organization. This does not relieve the proposer from the requirement to
include, as part of their submission (via BAAT), subcontract proposals that do not include
proprietary pricing information (rates, factors, etc.).
A Rough Order of Magnitude (ROM), or similar budgetary estimate, is not considered a fully
qualified subcontract cost proposal submission. Inclusion of a ROM, or similar budgetary
estimate, may result in the full proposal being deemed non-compliant or evaluation ratings
may be lowered;
\section{Cost-Sharing}
The amount of any industry cost-sharing (the source and nature of any proposed cost-sharing
should be discussed in the narrative portion of the cost volume); AND
\section{Fundamental Research}
Written justification required per Section II.B, “Fundamental Research,” pertaining to prime
and/or subcontracted effort being considered Contracted Fundamental Research.

