
%%%%%%%%% PROPOSAL -- 15 pages (including Results from Prior NSF Support)

\required{Project Description}
%\begin{figure*}[!t]
%\centering
%\includegraphics[width=7in]{cnnrnn}
% where an .eps filename suffix will be assumed under latex, 
% and a .pdf suffix will be assumed for pdflatex; or what has been declared
% via \DeclareGraphicsExtensions.
%\caption{Block diagram of CNN+RNN neural network.}
%\label{fig_cnnrnn}
%\end{figure*}

% From the NSF Grants Proposal Guide:
% "The Project Description should provide a clear statement of the work 
% to be undertaken and must include the objectives for the period of the 
% proposed work and expected significance; the relationship of this work
% to the present state of knowledge in the field, as well as to work in 
% progress by the PI under other support.
%
% The Project Description should outline the general plan of work, 
% including the broad design of activities to be undertaken, and, 
% where appropriate, provide a clear description of experimental 
% methods and procedures. Proposers should address what they 
% want to do, why they want to do it, how they plan to do it, how 
% they will know if they succeed, and what benefits could accrue
% if the project is successful. The project activities may be based
% on previously established and/or innovative methods and approaches,
% but in either case must be well justified. These issues apply to 
% both the technical aspects of the proposal and the way in which
% the project may make broader contributions."
Magnetic sensors
Applications
        compass
        MEG
        space
Techniques
        hall
        gmr
        fluxgate
        squid
        serf
        lorentz
Drawbacks
Nature
        CRY
        magnetosomes
Biomimetics
        mimic cilia (cite flow meter)
MEMS transducers
     piezo cantilevers
     magnetite nanoparticles
Plan (one year):
     Theory
     Multiphysics simulation
     Fabrication plan
%The Project Description is expected to be brief (five to eight pages) and include clear statements as to why this project is appropriate for EAGER funding, including why it does not “fit” into existing programs and why it is a “good fit” for EAGER. Note this proposal preparation instruction deviates from the standard proposal preparation instructions contained in this Guide; EAGER proposals must otherwise be compliant with the GPG.The EAGER funding mechanism may be used to support exploratory work in its early stages on untested, but potentially transformative, research ideas or approaches. This work may be considered especially "high risk-high payoff" in the sense that it, for example, involves radically different approaches, applies new expertise, or engages novel disciplinary or interdisciplinary perspectives.
\required{Broader Impacts}
% discuss the broader impacts of the prior NSF supported award
% (d) a listing of the publications resulting from the NSF award. 
% A complete bibliographic citation for each publication must be
% provided either in this section or in the References Cited section 
% of the proposal; if none, state "No publications were produced under this award."
% Due to space limitations, it is often advisable to use citations rather
% than putting the titles of the publications in the body of this section
% (e) evidence of research products and their availability, including, 
% but not limited to: data, publications, samples, physical collections, 
% software, and models, as described in any Data Management Plan
% (f) if the proposal is for renewed support, a description of the relation
% of the completed work to the proposed work.

% e.g.: "My prior grant, "Uses of Coffee in Mathematical Research" (DMS-0123456, 
% $100,000, 2012-2015), resulted in 3 papers [1],[2],[3], demonstrating..."

% if requesting postdoctoral research salary, a separate supplemental 1-page document
% called "Postdoc Mentoring Plan" will be required 
