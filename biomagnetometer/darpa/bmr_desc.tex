
\section{Statement of Work (SOW)}
In plain English, clearly define the technical tasks/subtasks to be performed, their durations,
and dependencies among them. The SOW page length will depend on the amount of effort.
For each task/subtask, provide:
1. A general description of the objective (for each defined task/activity);
2. A detailed description of the approach to be taken to accomplish each defined
task/activity;
3. Identification of the primary organization responsible for task execution (prime,
sub, team member, by name, etc.);
4. The completion criteria for each task/activity - a product, event or milestone that
defines its completion;
5. Define all deliverables (reporting, data, reports, software, etc.) to be provided to
the Government in support of the proposed research tasks/activities; See Section
I.F.
6. Identify whether government-furnished equipment is requested (see I.G) and, if
so, the required quantity and delivery schedule;
7. Clearly identify any Risk Reduction tasks (see II.D, below); AND
8. Clearly identify any tasks/subtasks (prime or subcontracted) that will be
accomplished on-campus at a university.
Note: Each program phase must be separately defined in the SOW. Include a SOW for each
subcontractor and/or consultant in the Cost Proposal Volume. Do not include any proprietary
information in the SOW(s).
\section{Innovative Claims}
Succinctly describe the uniqueness and benefits of the proposed approach relative to current
state-of-art alternate approaches.
\section{Detailed Technical Approach}
previous work on magnetometers

\cite{lenz2006magnetic} Magnetic sensors and their applications

\cite{shah2013compact} A compact, high performance atomic magnetometer for biomedical applications

\cite{shen2008design} The design, fabrication and evaluation of a MEMS PZT cantilever with an integrated Si proof mass for vibration energy harvesting

\cite{coey2010magnetism} Magnetism and magnetic materials

\cite{sasada2014fundamental} Fundamental mode orthogonal fluxgate gradiometer

\cite{sasada2002orthogonal} Orthogonal fluxgate mechanism operated with dc biased excitation

\cite{uchiyama2014highly} Highly sensitive CMOS magnetoimpedance sensor using miniature multi-core head based on amorphous wire

magneto reception in nature

\cite{johnsen2005physics} The physics and neurobiology of magnetoreception

\cite{dodson2013radical} A radical sense of direction: signalling and mechanism in cryptochrome magnetoreception

\cite{hanzlik2002pulsed} Pulsed-field-remanence measurements on individual magnetotactic bacteria

\cite{eder2012magnetic} Magnetic characterization of isolated candidate vertebrate magnetoreceptor cells

\cite{kirschvink2001magnetite} Magnetite-based magnetoreception


our approach

\cite{van2006resonant} Resonant frequencies of a rectangular cantilever beam immersed in a fluid

\cite{arnold2009permanent} mag mems

\cite{alfadhel2014magnetic} A magnetic nanocomposite for biomimetic flow sensing

\cite{sinha201627} 27 pT Silicon Nitride MEMS Magnetometer for Brain Imaging

\cite{kyynarainen20083d} 3D micromechanical compass

\cite{kumar2015ultra} Ultra sensitive Lorentz force MEMS magnetometer with pico-tesla limit of detection

\cite{thompson2009parametrically} Parametrically amplified MEMS magnetometer

\cite{levinzon2004fundamental} Fundamental noise limit of piezoelectric accelerometer




This is the centerpiece of the proposal and should provide a detailed description of the
proposed technology, including analysis and modeling where available, to substantiate the
innovative claims of Section II.B.

This section must include a proposed milestone table and
performance objectives, by phase, similar to Table 1 of this BAA. Proposals should clearly
explain the technical approach that will be employed to meet or exceed each program metric
and provide ample justification as to why the approach is feasible. Where applicable, analysis
should include concise performance budget tables, e.g. for contributory error or power
budget elements.

\section{Risk Analysis and Mitigation Plan}
Identify the major technical and programmatic risks in the program. Include a risk matrix.
For each risk, assign a probability of occurrence on a scale of 1-10, where 10 indicates a high
likelihood that the risk will impact program success, as well as an assessment of impact, also
on a scale of 1-10, where 10 indicates that this risk would maximally limit the program from
delivering prototypes on schedule or meeting performance objectives. For each item with
total risk (likelihood × impact) exceeding 40, include a plan for mitigating the risk and
assessing risk reduction.
Where necessary, parallel risk reduction tasks may be proposed, e.g. concurrent development
of redundant techniques or components. The proposal must differentiate the primary
technical path from risk reduction tasks, which should be uniquely identified in the SOW and
separately costed as optional tasks in Volume II.
\section{Schedule and Milestones}
Include a high-level Gantt chart outlining major technical tasks and measureable milestones
by phase. At a minimum, the schedule should include each SOW task of Volume 1, Section
II.A. Where risk reduction tasks are proposed, the schedule should include a milestone for
assessment and removal of redundant tasks.
\section{Test Plan}
Describe how compliance with the proposed metrics and milestones will be demonstrated in
each phase of the program. The test plan should be structured so that compliant performance
can be verified prior to delivery of hardware for government test and evaluation.
\section{Results and Technology Transfer}
Description of the results, products, transferable technology, and expected technology
transfer. This should also address mitigation of life-cycle and sustainment risks associated
with transitioning intellectual property for U.S. military applications, if applicable. See also
Section IV.B.10, “Intellectual Property.”
\section{Ongoing Research}
Comparison with other ongoing research indicating advantages and disadvantages of the
proposed effort.
\section{Proposer Accomplishments}
Discussion of proposer’s previous accomplishments and work in closely related research
areas. In this section, also include any ongoing research projects or pending proposal activity
that technically overlaps with the proposed effort, including funding source, administrative
point of contact, and the program management plan for combining and de-conflicting the
efforts.
\section{Facilities}
Description of the facilities that will be used for the proposed effort.
\section{Teaming}
Description of the formal teaming agreements that are required to execute this program.
Describe the programmatic relationship between investigators and the rationale for choosing
this teaming strategy. Present a coherent organization chart and integrated management
strategy for the program team. For each person, indicate: (1) name, (2) affiliation, (3)
abbreviated listing of all technical area tasks they will work on with roles, responsibilities,
and percent time indicated, (4) discussion of the proposers’ previous accomplishments,
relevant expertise and/or unique capabilities.
