\required{Terahertz Resonators for Sensitive Subsea Hydrocarbon Detection}

\required{Objectives}
The primary objectives of this research are to develop, characterize, and deploy an in-situ monitoring of drilling mud compostion and seawater composition for both kick detection and wellhead leak detection.

\required{Overview}

Offshore drilling suffers from a number of environmental risks pertaining to the uncontrolled release of hydro carbons into the ocean \cite{vinnem2014lessons, skogdalen2012quantitative}. Among these are wellhead blowouts, and wellhead leaks. Blowouts are often preceded by 'kicks', an outflow of hydrocarbons due to a sudden change in the pressure differential between the drilling mud and the pressure in the formation. Sensors which are able to detect minute changes in hydrocarbon concentration at the seafloor could function as an early warning system. Additionally, detecting changes in the composition of the drilling mud in the pipe at the seafloor could pinpoint precursors to kicks.

Many sensor modalities exist on drill rigs today \cite{petrowiki}. Flow sensors monitor drilling mud using whisker or proximity counting sensors. Diaphrams serve as pressure sensors in the pumps. Drill monitor measure RPM and torque. Pit monitors measure depth through ultrasonic echolocation. A combination gas trap and flame ionization detector is often used to determine hydrocarbon gas composition. Presently, kick indicators include decreases in pump pressure, increases in flow rate, increases in pit volume, and changes in mud composition. An open issue, highly pertinent to kick detection, is a sensor that could track compositional changes in the seawater and also in the drilling mud for kick detection.

In the terahetz regime, both saltwater \cite{xu20070} and petrochemicals \cite{wilt1998determination,jin2008analysis} are susceptible to low energy events caused by inter and intra molecular hydrogen bonds. Fourier-transform infrared spectroscopy is presently used for analyzing hydrocarbons, because they have resonances at these frequencies \cite{fodor1996analysis,wilt1998determination}. Time-domain spectroscopy has been demonstrated recently for imaging petrochemical and water containing cores. Another simpler, measurement technique for fluids in-situ is surface plasmon resonance (SPR), which operates at terahertz frequencies, and supports a simple microfluidic sensor form factor \cite{shibayama2016surface}. SPR operates by transforming an beam in the terahertz regime into a wave of charge (plasmon) on a negative-permittivity metal that supports a field that propagates into the media to be measured. The reflected beam angle, or its intensity, serve as an indicator of the composition of the media \cite{raether1988surface}. Such a sensor, mounted on the drill pipe, could sample the drilling mud on-line and in-situ for highly sensitive mud compostion tracking. In addition, it could measure seawater near the wellhead for a sensitive leak detector.

The investigators will develop an SPR sensor for in-situ monitoring of subtle changes in mud and seawater composition. The first stage will be computer-aided design, through simulation, followed by fabrication and characterization in the lab, and finally, real-world testing on a drill rig.

\required{Outputs and Stakeholder Involvement}

The key output will be an SPR sensor for early kick detection. Stakeholders will be involved at the time of real-word testing.

\required{Key Personnel}

\begin{description}
\item[Dr. Joaquin Casanova]
\item[Dr. Jenshan Lin]
\item[]
\end{description}
