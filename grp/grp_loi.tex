
%%%%%%%%% SUMMARY -- 1 page, third person
% e.g:  "The PI will prove" not "I will prove"
\required{Terahertz Resonators for Sensitive Subsea Hydrocarbon Detection}

\required{Objectives}
The primary objectives of this research are

\required{Overview}

Offshore drilling suffers from a number of environmental risks pertaining to the uncontrolled release of hydro carbons into the ocean. Among these are wellhead blowouts, and wellhead leaks. Blowouts are often preceded by 'kicks', an outflow of hydrocarbons due to a sudden change in the pressure differential between the drilling mud and the pressure in the formation. Sensors which are able to detect minute changes in hydrocarbon concentration at the seafloor could function as an early warning system. Additionally, detecting changes in the composition of the drilling mud in the pipe at the seafloor could pinpoint precursors to kicks.

Many sensor modalities exist on drill rigs today. ...

Fourier-transform infrared spectroscopy is presently used for analyzing hydrocarbons, because of they have vibrational resonances at these frequencies. Another simpler, measurement technique for fluids in-situ is surface plasmon resonance, which opperates at terahertz frequencies...

Seawater effect

How sensitive? Composition and character

This research hopes to develop terahertz SPR sensors for determining subtle changes in seawater and drilling mud

\required{Outputs and Stakeholder involvement}

\required{Key Personnel}

\begin{description}
\item[Dr. Joaquin Casanova]
\item[Dr. Jenshan Lin]
\item[]
\end{description}
