
\required{Project Description}
\section{Introduction}

The goal of this project is to develop and test a multimodal, noninvasive, networked beehive health monitor, that would be distributed among apiculturists to aggregate data on beehive health in real-time.

Pollinators perform an important ecological service in the reproduction of 70\% of plant species. Of pollinators, honeybees (\textit{Apis mellifera}) are special in their gathering of large amounts of pollen for food, in the process transferring pollen between flowers. Colony Collapse Disorder (CCD), named in 2007, reflects part of a decades long decline in the population of managed honeybees \cite{spivak2010plight}. CCD is characterized by a sudden colony die-off where worker bees fly off and abandon the hive. This is of particular concern, as agricultural production of pollinator dependent crops is increasing to meet the food needs of a growing human population.The USDA’s “Report on the National Stakeholders Conference on Honeybee Health” provides a basis for understanding the root causes of these population declines \cite{national2012report}. Several causes have worked in concert. Invasive \textit{Varroa destructor} mites are the primary concern, detrimental on its own and known to amplify viruses' effect, including Deformed Wing Virus. Lack of forage  leads to decreased food stores, and suppression of bee gut microbiota. Depressed immunity also results from chronic sublethal pesticide exposure. The changes result in physically detectable changes to the bees appearance and foraging behavior, as well as the hive as a whole, in terms of swarming, food stores and population. 


According to the “Pollinator Research Action Plan” goals \cite{national2015report}, the aim of US government-sponsored research is to reduce honeybee overwintering losses to no more than 15\%. Part of this overall goal is to establish a baseline for the current state of honeybee health. While there are honeybee censuses \cite{vanengelsdorp2011survey}, there exists no monitoring system for the health status of managed hives in the US. Such a network of standardized beehive monitors would enable researchers to mine data associated with crop and apiary management practices with measures of colony fitness.  
murphy2015big
Our project’s aim is to design, build, and test a noninvasive, wireless, multimodal beehive monitor that could be used nationwide. In particular, our system would implement visual, audio, and weight sensors to monitor bee health and behavior, as well as hive food content, without having to disturb the hive. As such, apiculturists could install the station easily. With 3G connection, these stations could upload data to a centralized database, enabling real-time monitoring of trends in beehive health. This could form the basis for a network similar to the weather station network of WeatherUnderground. In this way researchers could understand variation of colony losses over space, time, weather, and management practices.
Sensors have been used to monitor beehives in previous works. In a similar vein, researchers in Ireland have combined visual, infrared, load cells, particulate monitors, and gas analyzers in the b+WSN and 3b projects \cite{murphy2015big, murphy2015b+wsn}. Beelab is an open source Canadian project, providing apiculturists with a highly simplified kit for hive data collection \cite{phillips2014testing}. So far, there is no US based project with these goals. Furthermore, previous works are invasive to the hive to collect meaningful data – placing a night-vision camera inside the hive may produce useful information,  but would disturb the hive and would have reduced acceptance among stakeholders.

It's clear that several sensing modalities can be useful for monitoring hive health, and be non invasive. Researchers have implemented video and computer vision systems to monitor bees outside of the hive with success \cite{azarcoya2014automatic,steen2011portable}. And while the presence or lack of certain miticides might be most indicative of CCD, increased bee head size and asymmetry are also indicators \cite{speybroeck2010weighing} that can be detected with computer vision. DWV can be detected visually as well \cite{de2010deformed}. Sound is an indicator of hive status as well. The queen emits a particular sound prior to a swarm (“piping”), and this has been detected in previous works using microphones \cite{ferrari2008monitoring, eren1997electronic}. Hive weight can indicate activity and food stores, and monitoring efforts have been made in the past \cite{meikle2008within}, requiring load cells. Such a measurement system would allow noninvasive monitoring of the colony's food. However , load cells are cumbersome to install, so an alternative will be devloped, expecting a technique known as time-domain reflectometry, in which a pair of electrodes on the surface of the hive propagate a pulse, whose travel time serves as an indication of the hive's dielectric constant \cite{dalton1984time}. The higher the dielectric constant, the longer the travel time. The dielectric constant of honey is significantly different than that of air or wood \cite{guo2010sugar}, so hive honey content can be monitored in this way.

By combining computer vision, audio monitoring, and weight/honey measurement,  a hive's health status can be well described. Further, with access to wireless networks and solar power, and meta-data regarding management practices, this information can be monitored remotely and aggregated for analysis and prediction. Keeping the system simple, inexpensive, and noninvasive will increase acceptance among stakeholders.

The specific objectives of the proposed research are to develop a system are to alert farmers to health problems with their hive, through visual, audio, and electromagnetic sensors, and to network the sensors to permit aggregation of data on hive health for future analysis. These objectives meet the USDA's priority of elucidation of individual or interacting factors that affect pollinator populations that will lead to the development of novel tools and technologies to mitigate their losses. The proposal largely depends on established techniques, but this seed grant will allow us to establish a program of beehive sensor research. Potential future outcomes include novel sensing modalities, such as electromagnetic honey content and pesticide concentration monitoring.

\section{Approach}

The planned research is comprised of five main tasks: computer vision algorithm development, audio sensor design, electromagnetic sensor design, sensor integration and programming, and field implementation.

\subsection{Computer Vision Algorithm Development}

In order to provide targeted data, images of the hive entrance need to meet certain requirements. Images must only be acquired when workers are present, and workers' features need to be extracted. Thus, a program will be developed that takes a video stream and captures images when bees are present, as in \cite{campbell2008video}. The first part of the CV algorithm must simply detect the presence of bees at the hive entrance, through simple thresholding (dark bees are easily distinguished against the white paint of the hive entrance) or more advanced techniques such as Support Vector Machines trained on bee features \cite{azarcoya2014automatic}. Then the individual bees will be tracked frame-to-frame using a method such as optical flow \cite{horn1981determining} or nodal analysis as in \cite{campbell2008video}. Finally, the individual bees shapes must be determined and reduced to a set of shape parameters, as with Hough circle fitting \cite{yuen1990comparative}, for exampl. Additional machine learning algorithms will be implemented and trained using human experts to distinguish healthy from diseased bees. The computer vision algorithm will be developed using test images and existing video on a desktop computer.

\subsection{Audio Processor Design}

A microphone on the side of the hive will pick up audio and amplify sounds produced by the bees, after which it will  be digitized and processed. The microphone/amplifier/digitization circuit will be designed and constructed. Audio frequencies below 1 kHz are sufficient for detection of swarms. A software algorithm that analyzes the spectral content will be developed that pinpoints swarming events following the techniques described in \cite{ferrari2008monitoring}. Evidence also suggests that queens can be detected; even though their sound is siilar to workers, they generate a stronger tone between 400 and 550 Hz \cite{eren1997electronic}. Potentially, there is a link between hive sounds and exposure to airborne toxins \cite{bromenshenk2009honey}

\subsection{Honey Content Sensor Design}
Monitoring the honey content will be done in two ways: first, through measurement of the hive weight \cite{meikle2008within}, a standard technique, and second, through time domain reflectometry, a novel technique in this application. TDR has been applied extensively in measurement of soil water content, such as with the commercially available soil probes from Acclima \cite{blonquist2005time, schwartz2016evaluation}, due to the high dielectric of water relative to dry soil. Such a commercial sensor could be easily repurposed and attached to the exterior of the hive for purpose of hive monitoring, and calibrated against the weight of the hive. A hive weight monitor will be designed to support the high weights (>200lb) of a full hive, and measure small changes with time. This can be done using off-the-shelf load cells and amplifiers. The signal will be digitized and logged and used as a reference against the electromagnetic sensor.

\subsection{Sensor Integration}

The three sensing systems will be integrated into one datalogging system using a low-power embedded computer, such as the Raspberry Pi \cite{upton2014raspberry}, which would also implement the CV algorithm. This computer would perform the image processing and log all data. In addition, it would serve the data to a web interface and central database that will be designed and programmed for this task. 

\subsection{Field Implementation and Controlled Expermients}

Finally, the sensors and embedded computer will be installed on a test hive. A weatherproof housing will be designed and fabricated.  The system will be powered by solar cells. A 3G connection will allow wireless data upload. The system will be monitored and debugged during the following 3 months. At least 10 hives will be divided into different treatment groups. Several relevant treatment variables could be controlled
artificially to evaluate the system’s efficacy, including presence of a queen, and Varroa and small hive beetle infestations, etc. Sensor measurements (honey content, bee morphological measurements, audio spectra) can be tested statistically for significance with regard to these explanatory variables indicative of hive health.

\section{Timeline}

The project will be completed over 18 months. Tasks 1-4 will be performed during May 2017 through August 2017. Task 4 will be completed in the 3 months following 1-3. Task 5, field implementation and testing, will follow, for a duration of 3 months. Finally, the system should be ready for a field experiment in Spring of 2018, for nine months, after which data will be analyzed and papers written for 3 months.
\section{Pitfalls}

There are several potential difficulties with this project. First, computer vision algorithm development is challenging, and depends largely on the quality of the test dataset. Gathering enough extant test video of beehive entrances, and formulating an effective feature extraction and analysis algorithm, may prove more time-consuming than anticipated. Second, the required sensitivity of the microphone and load cells are as yet unknown, and method of dynamically adjusting both for environmental factors (ie, loud noises and hive shifts due to maintenance) will need to be developed. Third, the field implementation will need to be kept highly power efficient to support such processor-intensive activities, such as computer vision, as well as wireless networking. At worst it may require  supplemental power, perhaps instead relying on power-over-ethernet. Finally, the system cost should be kept low enough that it would be readily accepted by apiculturists, so that data could be aggregated.
