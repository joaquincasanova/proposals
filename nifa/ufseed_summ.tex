
%%%%%%%%% SUMMARY -- 1 page, third person
% e.g:  "The PI will prove" not "I will prove"

\required{Abstract}

Pollinators, such as honeybees, bumblebees, hummingbirds, wasps, and butterflies, serve an important ecological function. They provide pollination services to many plants, thus facilitating plant reproduction. Honey bees (\textit{Apis mellifera}) in particular pollinate many of the food crops used by humans and are responsible for billions of dollars of added crop value.The population of managed honey bees has been under constant threat the last decade, with many parts of the world experiencing colony loss rates exceeding 30\% yearly. The stressors impacting managed colonies result in physically detectable changes to the bees’ appearance, foraging behavior, colony strength parameters, etc. The causes of and solutions for this phenomenon are still active areas of research. Elevated colony losses are a multifactorial problem, with several potential causes working in concert, including diseases (viruses, bacteria and fungi), pests (the most notable of which is \textit{Varroa destructor}, a parasitic mite), pesticides, poor nutrition, and loss of habitat.

To address this concern, the US government established a set of research goals and recovery actions in the Pollinator Research Action Plan. Research goals included establishing pollinator health baselines, assessing environmental stressors, restoring habitat, supporting land managers and beekeepers (including the development of best management practices), and pollinator monitoring and study. Part of our overall goal is to establish a baseline for the existing state of honey bee health. Currently, there exists no good system for monitoring the health of managed colonies in the US. Our project’s aim is to design, build, and test a noninvasive, wireless, multimodal beehive monitor that could be used to monitor colony health nationally. Such a system, if widely adopted, would allow stakeholders to monitor their own hives, and allow researchers to collect data easily and remotely. Combined with additional geographical, and land management data (ie, pesticide use, proximity to crops, etc.), this data network would aid in highlighting causes of, and developing BMPs to mitigate the elevated colony losses.

An inexpensive, wirelessly connected beehive monitor could achieve widespread acceptance in the beekeeping industry. Several easily measured variables include: bee traffic at the hive entrance and the general shape of their bodies (a proxy for their health - measured through a camera mounted at the hive entrance; presence of a queen in the hive, general hive activity (measured as audio frequencies produced by the hive); and honey content (measured through the electromagnetic characteristics of the hive body). These metrics provide data on bee diseases (worker bee shape changes under certain disease stresses), colony propensity to swarm, and availability of food in- and outside the nest. Such sensors may be devised by repurposing existing, off-the-shelf sensors, and integrated them into a single system that transmits its data to a server wirelessly. This research aims to develop and test such a system, with hope of commercialization and widespread acceptance.
