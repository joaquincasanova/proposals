
%%%%%%%%% SUMMARY -- 1 page, third person
% e.g:  "The PI will prove" not "I will prove"

\required{Abstract}

Pollinators, such as honeybees (\textit{Apis mellifera}) and bumblebees, hummingbirds, wasps, and butterflies, serve an important ecological function: allowing plants to reproduce, and in so doing, producing the majority of food crops relied on by humans. Honeybees themselves are responsible for billions of dollars in crop value. The past decade has seen a sudden increase in honeybee winter losses, which has come to be known as Colony Collapse Disorder (CCD). The causes and solutions of this phenomenon are still an active area of research. It's a multifactorial problem, with several potential causes working in concert, among which are disease (Deformed Wing Virus), pests (\textit{Varroa destructor}), immunodepression by pesticides, and loss of honeybee habitat.

To address this concern, the US government established a set of research goals and recovery actions in the Polinator Research Action Plan. Research goals include establishing pollinator health baselines, assessing environmental stressors, restoring habitat, supporting land managers and beekeepers (including best manangement practices), and pollinator monitoring and study. The aim of this project is do develop an inexpensive sensor system for monitoring hive health. Such a system, if widely adopted, would allow stakeholders to monitoring their own hives, and allow researchers to collect data nationwide easily. Combined with additional geographical, and land management data (ie, pesticides, herbicides, nearby crops), this data network would aid in highlighting causes of, and developing BMPs to mitigate, CCD.

An inexpensive, wirelessly connected beehive monitor could achieve widespread acceptance. Several easily measured variables include: bee traffic and shape (measured through a camera at the high entrance; presence of a queen and hive activity (measured through the audio frequencies produced by the hive); and honey content (measured through the electromagnetic characteristics of the hive body). These metrics address disease, as worker bee shape changes under certain disease stresses; potential for swarming; and availability of food, and indirectly habitat, through honey content. All three of these sensors may be devised by reprposing off-the-shelf sensors, and integrated into a single system with transmits it's data to a server wirelessly. This research aims to develop and test such a system, with hope of commercialization and widespread acceptance.
