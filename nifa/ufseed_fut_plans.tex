%%%%%%%%% DATA MANAGEMENT PLAN -- 2 pages
\required{Continued Support and Commercial Potential}

This project may be viewed as a preliminary study - significant advancements could be made as we increase the scale and advance then sensors. Once apiculturists are widely using such a sensor platform, this bacomes a very useful tool. Aggregating hive data, along with weather condition, location, and nearb pesticide and herbicide usage, can elucidate unknown correlations that will highlight causes of CCD. Additionally, further advances in sensors may be developed and implemented, such as bee tracking in the field with lasers \cite{carlsten2011field} or infra-red monitoring of bee population \cite{shaw2011long}. Potential future sources of funding include:

\begin{itemize}
\item USDA NIFA -- AFRI Food Security Challenge Area
\item Fresh From Florida -- Specialty Crop Grant
\item DOI -- Competitive State Wildlife Grant Program
\end{itemize}

These types of sensor systems would become inexpensive on a large scale so it is a good candidate for commercialization, and there is certainly a market for it among apiculturists and apicultural researchers.
