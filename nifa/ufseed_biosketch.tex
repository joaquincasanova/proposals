
%%%%%%%%% BIOGRAPHICAL SKETCH -- 2 pages

\required{Biographical Sketch: Joaquin Casanova}

% Your Bio should be divided into the following sections
% It is not required to include the parenthecital letters preceding
\required{(a) Professional Preparation}
\begin{description}
\item[Electrical Engineering PhD,] UF, Gainesville, May 2010
\item[Agricultural and Biological Engineering ME,] UF, Gainesville, December 2007
\item[Agricultural and Biological Engineering BS,] UF, Gainesville, December 2006
\end{description}

\required{(b) Appointments}
\begin{description}
\item[Research Assistant Professor,] University of Florida, August 2016-present
\item[Senior Engineer,] University of Florida, November 2013-June 2016
\item[Research Engineer,] USDA,	May 2010-October 2013
\end{description}

\required{(c) Products}
30 technical publications in peer-reviewed journals and conference proceedings. 2 patents awarded.


Most closely related to the proposed project:	
\begin{enumerate}
	
\item Schwartz, R. C., \textbf{Casanova, J. J.}, Bell, J. M., \& Evett, S. R. (2014). A reevaluation of time domain reflectometry propagation time determination in soils. Vadose Zone Journal, 13(1).	

\item \textbf{Casanova, J. J.}, Schwartz, R. C., \& Evett, S. R. (2014). Design and field tests of a directly coupled waveguide-on-access-tube soil water sensor.Applied Engineering in Agriculture, 30(1), 105-112.

\item \textbf{Casanova, J. J.}, O'Shaughnessy, S. A., Evett, S. R., \& Rush, C. M. (2014). Development of a wireless computer vision instrument to detect biotic stress in wheat. Sensors, 14(9), 17753-17769.

\item \textbf{Casanova, J. J.}, O’Shaughnessy, S., \& Evett, S. (2013, November). Wireless computer vision system for crop stress detection. In ASA-CSSA-SSSA Annual Meeting Abstracts (p. 123). ASA-CSSA-SSSA Annual Meeting Abstracts. Session 196-7.

\item \textbf{Casanova, J. J.}, Evett, S. R., \& Schwartz, R. C. (2012). Design and field tests of an access-tube soil water sensor. Applied Engineering in Agriculture,28(4), 603-610.

\item \textbf{Casanova, J. J.}, Evett, S. R., \& Schwartz, R. C. (2012). Design of access-tube TDR sensor for soil water content: Testing. Sensors Journal, IEEE,12(6), 2064-2070.

\item \textbf{Casanova, J. J.}, Evett, S. R., \& Schwartz, R. C. (2012). Design of access-tube TDR sensor for soil water content: Theory. Sensors Journal, IEEE,12(6), 1979-1986.

\item Garnica, J., \textbf{Casanova, J. J.}, \& Lin, J. (2011, May). High efficiency midrange wireless power transfer system. In Microwave Workshop Series on Innovative Wireless Power Transmission: Technologies, Systems, and Applications (IMWS), 2011 IEEE MTT-S International (pp. 73-76). IEEE.

\item \textbf{Casanova, J. J.}, Taylor, J. A., \& Lin, J. (2010). Design of a 3-D fractal heatsink antenna. Antennas and Wireless Propagation Letters, IEEE, 9, 1061-1064.

\item Low, Z. N., \textbf{Casanova, J. J.}, Maier, P. H., Taylor, J. A., Chinga, R. A., \& Lin, J. (2010). Method of load/fault detection for loosely coupled planar wireless power transfer system with power delivery tracking. Industrial Electronics, IEEE Transactions on, 57(4), 1478-1486.

\item \textbf{Casanova, J. J.}, Low, Z. N., \& Lin, J. (2009). Design and optimization of a class-E amplifier for a loosely coupled planar wireless power system.Circuits and Systems II: Express Briefs, IEEE Transactions on, 56(11), 830-834.

\item \textbf{Casanova, J. J.}, Low, Z. N., \& Lin, J. (2009). A loosely coupled planar wireless power system for multiple receivers. Industrial Electronics, IEEE Transactions on, 56(8), 3060-3068.

\item \textbf{Casanova, J. J.}, Judge, J., \& Jang, M. (2007). Modeling transmission of microwaves through dynamic vegetation. Geoscience and Remote Sensing, IEEE Transactions on, 45(10), 3145-3149.
\end{enumerate}
\required{(d) Synergistic Activities}

\begin{enumerate}
\item \textbf{Main Activities} Dr. Casanova is a research assistant professor in the Department of Electrical and Computer Engineering at the University of Florida. His main research activities are electromagnetic sensors, instrumentation design, and machine intelligence applications. Previously he did research with the USDA in these areas and developed chemistry instrumentation for UF’s Chemistry Department.
\item \textbf{Professional Membership}
\begin{itemize}
\item 2004-–present Member American Society of Agricultural and Biological Engineers (ASABE)

\item 2006-–present Member Institute of Electrical and Electronics Engineers (IEEE)
\end{itemize}
\end{enumerate}



