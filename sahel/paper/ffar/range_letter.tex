
%%%%%%%%% MASTER -- compiles the 6 sections

\documentclass[11pt,letterpaper]{article}

%%%%%%%%%%%%%%%%%%%%%%%%%%%%%%%%%%%%%%%%%%%%%%%%%%%%%%%%%%%%%%%%%%%%%%%%%
\pagestyle{plain}                                                      %%
%%%%%%%%%% EXACT 1in MARGINS %%%%%%%                                   %%
\setlength{\textwidth}{6.5in}     %%                                   %%
\setlength{\oddsidemargin}{0in}   %% (It is recommended that you       %%
\setlength{\evensidemargin}{0in}  %%  not change these parameters,     %%
\setlength{\textheight}{8.5in}    %%  at the risk of having your       %%
\setlength{\topmargin}{0in}       %%  proposal dismissed on the basis  %%
\setlength{\headheight}{0in}      %%  of incorrect formatting!!!)      %%
\setlength{\headsep}{0in}         %%                                   %%
\setlength{\footskip}{.5in}       %%                                   %%
%%%%%%%%%%%%%%%%%%%%%%%%%%%%%%%%%%%%                                   %%
\newcommand{\required}[1]{\section*{\hfil #1\hfil}}                    %%
\renewcommand{\refname}{\hfil References Cited\hfil}    %%
\bibliographystyle{plain}
                                              %%
%%%%%%%%%%%%%%%%%%%%%%%%%%%%%%%%%%%%%%%%%%%%%%%%%%%%%%%%%%%%%%%%%%%%%%%%%

%PUT YOUR MACROS HERE

\usepackage{amsmath}
\usepackage{amssymb}
\usepackage{cite} 


\pagestyle{empty}
%\includeonly{NSFsumm}

\begin{document}

\begin{description}
\item [PI:] Joaquin Casanova
\item [Department:] Electrical and Computer Engineering
\item [Contact:] jcasa@ufl.edu, 352-294-2024
\item [Start Date:] 6/1/2016
\item [Target Area:] Enhancing Sustainable Farm Animal Productivity, Resilience, and Health
\end{description}
\required{Summary of Research Goals}

The Sahel region of Africa, a ecosystem south of the Sahara comprised of primarily savannah \cite{wwf}, is suffering from a mix of severe problems, ultimately affecting human well-being, local ecosystem health, and global climate. Prolonged drought in the region has led to declines in food production; when combined with overgrazing this has resulted in soil erosion can result from excessive patch grazing \cite{coughenour1991spatial}; this reduces the soil's ability to support vegetation when rains do arrive. Traditionally, in the Sahel region, herders were nomadic or semi-nomadic, bringing their herds to the north during the brief wet season, and south to the Niger Delta in the dry season \cite{coughenour1991spatial}. Now, there is intense competition for grazing in the relatively wet Niger Delta between these semi-nomads and sedentary farmers in the Delta. Ultimately, this region is suffering from the tragedy of the commons, where public resources are used until depletion. Greater coordination is needed between farmers and herders to avoid this problem \cite{breman1983rangeland}. The proposed research aims to provide farmers and herders a technological way to manage the grazing resources of the Sahel and avoid the tragedy of the commons.

To help solve this resource management issue, this research proposes developing, testing, and implementing a decision support system which can guide grazing patterns for farmers and herders for optimal economic and environmental outcomes. Similar software has been developed for northern Ethiopia \cite{dragan2003application}. By feeding multispectral data from  wireless sensors, GPS tracking of livestock, and satellite data, into models for flora, fauna, climate, and economics, any individual could predict the outcomes of their farming/grazing actions. 

Livestock are selective in what they graze on. Cattle's protein requirements vary over the year so grazing patterns show important patterns at small and large scales as they choose grass species higher in nitrogen \cite{coughenour1991spatial}. Addtionally, topographic features, such as water, lead cattle to congregate and overgraze. Thus, sensors should be capabale of relaying grazing patterns and metrics of plant health and diversity on both scales. UAVs with computer vision, to identify plant species and density, and GPS cattle tracking, for monitoring grazing \cite{wark2007transforming}, can handle the small scale variations. Additional local measurements could include microwave sensing of soil water and local weather stations. Satellite data (in visible and near-infrared) has been used to assess the health of rangelands in terms of vegetation health and nutritional content on large scales \cite{knox2012remote}. Sticking to simple sensors mounted on cattle and drones, and already available satellite data, provides data cheaply and effectively in an economically-stressed region. Feeding this data into vegetation/animal/economic models, similar to SPUR \cite{carlson1996comprehensive}, or soft computing techniques like genetic programming, can provide predictions for the courses of action which optimally benefit individuals and the ecological health of the region. Such software would be implemented in a smartphone app, as smartphone adoption is high in this region of Africa.

\required{Qualifications}
Dr. Casanova has worked primarily on agricultural sensors for most of his career, and is thus well qualified for this project. His work includes microwave remote sensing of soil moisture, and crop and microclimate modelling, at UF's Center for Remoe Sensing (from 2005-2007). He developed soil water sensors and computer vision sensors for disease/drought detection in wheat and cotton with the USDA (2010-2013). In 2015 he wrote computer vision and water loss estimating software for drone-based monitoring of grapevine health for a small vineyard in Oklahoma. Presently, he is a Research Assistant Professor in the Department of Electrical and Computer Engineering, researching low-field magnetic sensors and deep learning algorithms in biomedical applications, in addition to teaching Antenna Design and Radio Frequency Systems. He'd like his future career to continue in the agricultural sector, and is pursuing proposals related to beehive health, rangeland monitoring, and water use efficiency.   

\required{Alignment with FFAR Goals}

The proposed work is aligned with the FFAR goal of enhancing sustainable farm animal productivity, resilience, and health. The Sahel in particular is ignored by Western researchers, who may find it difficult to obtain government support for research in Africa.

\bibliography{./casanova_ieee}

\end{document}
\grid
\grid
