
%%%%%%%%% SUMMARY -- 1 page, third person
% e.g:  "The PI will prove" not "I will prove"

\required{Project Summary}
\required{Overview}
The Sahel region of Africa, a ecosystem south of the Sahara comprised of primarily savannah, is suffering from a mix of severe problems, ultimately affecting human well-being, local ecosystem health, and global climate. Prolonged drought in the region has led to declines in food production; when combined with overgrazing this has resulted in soil erosion can result from excessive patch grazing; this reduces the soil's ability to support vegetation when rains do arrive. Traditionally, in the Sahel region, herders were nomadic or semi-nomadic, bringing their herds to the north during the brief wet season, and south to the Niger Delta in the dry season. Now, there is intense competition for grazing in the relatively wet Niger Delta between these semi-nomads and sedentary farmers in the Delta. Ultimately, this region is suffering from the tragedy of the commons, where public resources are used until depletion. Greater coordination is needed between farmers and herders to avoid this problem, and greater understanding of the interactions between human decision making, agroecosystem dynamics, and the available cyberinfrastructure for monitoring. The proposed research aims to provide farmers and herders a technological way to understand and manage the grazing resources of the Sahel.

To help solve this resource management issue, this research proposes developing, testing, and implementing a decision support system which can guide grazing patterns for farmers and herders for optimal economic and environmental outcomes. By feeding multispectral data from  wireless sensors, GPS tracking of livestock, and satellite data, into models for flora, fauna, climate, and economics, any individual could predict the outcomes of their farming/grazing actions. 

Livestock are selective in what they graze on. Cattle's protein requirements vary over the year so grazing patterns show important patterns at small and large scales as they choose grass species higher in nitrogen. Addtionally, topographic features, such as water, lead cattle to congregate and overgraze. Thus, sensors should be capabale of relaying grazing patterns and metrics of plant health and diversity on both scales. UAVs with computer vision, to identify plant species and density, and GPS cattle tracking, for monitoring grazing, can handle the small scale variations. Additional local measurements could include microwave sensing of soil water and local weather stations. Satellite data (in visible and near-infrared) has been used to assess the health of rangelands in terms of vegetation health and nutritional content on large scales. Sticking to simple sensors mounted on cattle and drones, and already available satellite data, provides data cheaply and effectively in an economically-stressed region. Feeding this data into vegetation/animal/economic models, or soft computing techniques like genetic programming, can provide predictions for the courses of action which optimally benefit individuals and the ecological health of the region. Such software would be implemented in a smartphone app, as smartphone adoption is high in this region of Africa.

%The FEW systems must be conceptualized broadly,
%incorporating physical processes (such as built infrastructure and new technologies for more efficient resource
%utilization), natural processes (such as biogeochemical and hydrologic cycles), biological processes (such as
%agroecosystem structure and productivity), social/behavioral processes (such as decision making and governance),
%and cyber-components (such as sensing, networking, computation and visualization for decision-making and
%assessment). 
%proposals submitted to the
%INFEWS solicitation must define the FEW systems intended for study. The FEW system(s) description should identify the
%systems boundaries and the primary food and energy and water components that make up the integrated FEW system(s) of
%the study.
%Proposals submitted to the INFEWS program must demonstrate meaningful integration across
%disciplines to address the principal objectives outlined below and should go beyond existing approaches that can be addressed
%within the individual disciplines and usual core-program co-funded research opportunities at NSF and USDA/NIFA.
%Proposals involving international collaboration should clearly describe the work that will be accomplished by the entire team,
%including the international partners, and how the international partners’ efforts will be supported.
%This solicitation outlines three tracks of research: (1) FEW System Modeling; (2) Visualization and Decision support for Cyber-
%Human-Physical Systems at the FEW Nexus; and (3) Research to Enable Innovative Solutions
%Cyber-human-physical systems (CHPS) integrate decision making at different spatial and temporal scales with sensing, computation,
%and networking measurements of the social, natural, physical and built worlds. From this perspective, INFEWS represents CHPS on
%a grand scale that is tightly woven between the physical and the human fabrics. Each FEW system is a large CHPS with human
%interaction influencing system outcomes. Track 2 seeks to develop the core system science needed to understand the interactions
%between these diverse but closely coupled components that operate at multiple temporal and spatial scales.
\section{Intellectual Merit}
Integration across disciplines


underserved and underinvestigated region


will significantly advance our understanding of the food-energy-water system through quantitative, predictive and computational modeling


develop real-time, cyber-enabled interfaces that improve understanding of the behavior of FEW systems and increase decision support capability

provides innovative solutions to critical FEW systems problems


\section{Broader Impact}
Grow the scientific workforce capable of studying and managing the FEW system, through education and other professional development opportunities.


Increase attention to agroecological problems in the developing world


Participation and training of undergraduate and graduate students


Enhance educational resources of the Department
