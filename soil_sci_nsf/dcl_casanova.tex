
%%%%%%%%% MASTER -- compiles the 6 sections

\documentclass[11pt,letterpaper]{article}
\pagestyle{plain}                                                      %%
%%%%%%%%%% EXACT 1in MARGINS %%%%%%%                                   %%
\setlength{\textwidth}{6.5in}     %%                                   %%
\setlength{\oddsidemargin}{0in}   %% (It is recommended that you       %%
\setlength{\evensidemargin}{0in}  %%  not change these parameters,     %%
\setlength{\textheight}{8.5in}    %%  at the risk of having your       %%
\setlength{\topmargin}{0in}       %%  proposal dismissed on the basis  %%
\setlength{\headheight}{0in}      %%  of incorrect formatting!!!)      %%
\setlength{\headsep}{0in}         %%                                   %%
\setlength{\footskip}{.5in}       %%                                   %%
%%%%%%%%%%%%%%%%%%%%%%%%%%%%%%%%%%%%                                   %%

\renewcommand{\refname}{\hfil References Cited\hfil}                   %%
%PUT YOUR MACROS HER
\usepackage{cite}

\bibliographystyle{plain}                                              %%
%%%%%%%%%%%%%%%%%%%%%%%%%%%%%%%%%%%%%%%%%%%%%%%%%%%%%%%%%%%%%%%%%%%%%%%%%

\usepackage[pdftex]{graphicx}
\graphicspath{{../pdf/}{../jpeg/}}
\DeclareGraphicsExtensions{.pdf,.jpeg,.png}
\usepackage{amsmath}
\usepackage{array}
\usepackage{fixltx2e}
\usepackage{stfloats}
\usepackage{url}
\usepackage{amssymb}

\begin{document}
\section{Title}
Terahertz Sensors for In-Situ Soil Characterization and Imaging
\section{Directorates}
This research is proposed as an EAGER submission and goes under the ENG directorate.
\section{Proposed Research}

Depth-varying soil properties such as moisture, nutrient levels, organic matter, and texture are critical for efficient crop managment. Precision agriculture, in which parameters such as these are used for efficient application of irrigation, fertilizer, and pesticides and herbicides, relies on reliable measurement of these properties.  Additionally, plant root structure aids in crop modeling, used to predict how changes in crop management affect yield. Presently, there is a lack of sensors for in-situ measurement of these properties. This research proposes a new application of two sensing modalities that have distinct advantages of low power, small size, and ease of installation: surface plasmon resonance (SPR) in the near-infrared (NIR) and terahertz (THz) imaging using beamforming radar.

Current soil profile sensing techniques involve either taking samples from the field, for lab analysis of properties, or difficult installation in the field. For instance, time-domain reflectometry (TDR) \cite{topp1980electromagnetic} or capacitance measurement \cite{birchak1974high} both can determine soil moisture and electrical conductivity through electromeagnetic characterization of soil in the megahertz regime. However, to measure the soil profile, this requires digging a pit and installing sensors, then back-filling. Moreover, operation in this frequency ranges necessitates larger sensors and larger electrical components.

In the past decade, researchers have sought to examin soil properties using near-infrared spectroscopy, in which the reflectance or transmittance of a soil sample is measure over a range of NIR frequencies. This is in the THz regime and makes use of small, low power, optical components, and has been shown to measure soil moisture, carbon, nitrogen, and texture \cite{chang2001near}. Also in the THz regime is imaging radar, in which pulses are sent into the soil an their transit time is used to spatially characterize soil structure. While fundamentally similar to TDR, this technique can resolve much finer features, potentially even roots \cite{dworak2011application}. By steering the radar beam over a range of spatial locations, an underground image in the root zone can be taken. Using such high frequencies enables using very small (sub-millimeter) radar antennas.

This research proposes exploiting measurement at very high frequencies to characterize soil moisture, organic matter, nutrients, and root structure. The proposed techniques have the distinct advantages of small size, low power, and variety of measurements. The sensor would use surface plasmon resonance, in which the NIR pulse is guided into the soil medium through a thin metal layer with a negative refractive index on top of an optical wave guide \cite{shibayama2016surface}. This allows NIRS underground at a range of depths by a pencil-thin rod carrying multiple optical waveguides, directing NIR pulses to the desired locations. Such a sensor is easily installable by drilling a narrow hole. Another THz sensor will be investigated simultaneously: beamforming THz radar. An array of antennas on the surface of a narrow tube would allow a radar beam to be steered through a range of locations around the sensor and at various depths. The radar pulse return times will form an image of the cylindrical voluume around the sensor, characterizing the root structure an soil stratification. In both sensors, THz pulses are coupled to small electrogmagnetic structures along a thin cylinder. The proposed work has three parts: design of the electromagnetic structure of the sensors; design of the circuitry for applying the THz pulses to te structures; and design of the microfabrication process for their construction. Over the course of a three-year investigation, we plan to start with theorectiacal design using fundamental principles combined with computer simulation, move to prototype construction and laboratory testing, and end with field trials.

The researchers involved include PI Joaquin Casanova, with an established history in agricultural engineering and electromagnetic sensor design; Co-PI Changzhi Li, with a strong research history in high-frequency radar techniques and circuit design; and Co-PI YK Yoon, who has been long involved in design and fabrication of micro-scale sensors. 

\section{Researchers and Contact Information}
\begin{description}
\item{PI} - Joaquin Casanova - University of Florida - jcasa@ufl.edu - 352-246-9649
\item{Co-PI} - Changzhi Li - Texas Tech University - changzhi.li@ttu.edu - 806-834-8682
\item{Co-PI} - Yong-Kyu Yoon -  University of Florida - ykyoon@ufl.edu  - 352-392-5985
\end{description}

\bibliography{casanova_ieee}

  
  
\end{document}
