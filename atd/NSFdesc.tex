
%%%%%%%%% PROPOSAL -- 15 pages (including Results from Prior NSF Support)

\required{Project Description}

% From the NSF Grants Proposal Guide:
% "The Project Description should provide a clear statement of the work 
% to be undertaken and must include the objectives for the period of the 
% proposed work and expected significance; the relationship of this work
% to the present state of knowledge in the field, as well as to work in 
% progress by the PI under other support.
%
% The Project Description should outline the general plan of work, 
% including the broad design of activities to be undertaken, and, 
% where appropriate, provide a clear description of experimental 
% methods and procedures. Proposers should address what they 
% want to do, why they want to do it, how they plan to do it, how 
% they will know if they succeed, and what benefits could accrue
% if the project is successful. The project activities may be based
% on previously established and/or innovative methods and approaches,
% but in either case must be well justified. These issues apply to 
% both the technical aspects of the proposal and the way in which
% the project may make broader contributions."

\required{Broader Impacts}
% As in the project summary, broader impacts of the proposed work
% must be called out separately in the project description.  
% You may be able to give more specific examples, 
% or discuss how you've previously achieved these impacts.
% It should be similar, but not identical, to the Broader Impacts statement
% in the project summary.

\required{Results From Prior NSF Support}
% Must be fewer than 5 pages of the entire description document of 15 pages.
% This section refers to any prior or current  NSF funding support
% with a start date in the past five years.
% If you have no prior support, it is still recommended to include this
% section and just indicate "The PI has not previously been supported by the NSF".
%If you have more than one award, you need only report on the one award most
% closely related to this proposal.
% The following information must be provided in this section:
% (a) the NSF award number; amount and period of support
% (b) the title of the project
% (c) a summary of the results of the completed work, including
% accomplishments, supported by the award. 
% The results must be separately described under two distinct headings, 
\required{Intellectual Merit}
% discuss the intellectual merit of the prior NSF supported award
\required{Broader Impacts}
% discuss the broader impacts of the prior NSF supported award
% (d) a listing of the publications resulting from the NSF award. 
% A complete bibliographic citation for each publication must be
% provided either in this section or in the References Cited section 
% of the proposal; if none, state "No publications were produced under this award."
% Due to space limitations, it is often advisable to use citations rather
% than putting the titles of the publications in the body of this section
% (e) evidence of research products and their availability, including, 
% but not limited to: data, publications, samples, physical collections, 
% software, and models, as described in any Data Management Plan
% (f) if the proposal is for renewed support, a description of the relation
% of the completed work to the proposed work.

% e.g.: "My prior grant, "Uses of Coffee in Mathematical Research" (DMS-0123456, 
% $100,000, 2012-2015), resulted in 3 papers [1],[2],[3], demonstrating..."

% if requesting postdoctoral research salary, a separate supplemental 1-page document
% called "Postdoc Mentoring Plan" will be required 
